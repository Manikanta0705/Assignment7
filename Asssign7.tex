%%%%%%%%%%%%%%%%%%%%%%%%%%%%%%%%%%%%%%%%%%%%%%%%%%%%%%%%%%%%%%%
%
% Welcome to Overleaf --- just edit your LaTeX on the left,
% and we'll compile it for you on the right. If you open the
% 'Share' menu, you can invite other users to edit at the same
% time. See www.overleaf.com/learn for more info. Enjoy!
%
%%%%%%%%%%%%%%%%%%%%%%%%%%%%%%%%%%%%%%%%%%%%%%%%%%%%%%%%%%%%%%%

% Inbuilt themes in beamer
\documentclass{beamer}

%packages:
% \usepackage{tfrupee}
% \usepackage{amsmath}
% \usepackage{amssymb}
% \usepackage{gensymb}
% \usepackage{txfonts}

\def\inputGnumericTable{}

% \usepackage[latin1]{inputenc}                                 
% \usepackage{color}                                            
% \usepackage{array}                                            
% \usepackage{longtable}                                        
% \usepackage{calc}                                             
% \usepackage{multirow}                                         
% \usepackage{hhline}                                           
% \usepackage{ifthen}
% \usepackage{caption} 
% \captionsetup[table]{skip=3pt}  
 \providecommand{\pr}[1]{\ensuremath{\Pr\left(#1\right)}}
 \providecommand{\cbrak}[1]{\ensuremath{\left\{#1\right\}}}
% %\renewcommand{\thefigure}{\arabic{table}}
% \renewcommand{\thetable}{\arabic{table}}      

\setbeamertemplate{caption}[numbered]{}

\usepackage{enumitem}
\usepackage{tfrupee}
\usepackage{amsmath}
\usepackage{amssymb}
\usepackage{gensymb}
\usepackage{graphicx}
\usepackage{txfonts}

\def\inputGnumericTable{}

\usepackage[latin1]{inputenc}                                 
\usepackage{color}                                            
\usepackage{array}                                            
\usepackage{longtable}                                        
\usepackage{calc}                                             
\usepackage{multirow}                                         
\usepackage{hhline}                                           
\usepackage{ifthen}
\usepackage{caption} 
\captionsetup[table]{skip=3pt}  
\providecommand{\pr}[1]{\ensuremath{\Pr\left(#1\right)}}
\providecommand{\cbrak}[1]{\ensuremath{\left\{#1\right\}}}
\renewcommand{\thefigure}{\arabic{table}}
\renewcommand{\thetable}{\arabic{table}}   
\providecommand{\brak}[1]{\ensuremath{\left(#1\right)}}

% Theme choice:
\usetheme{CambridgeUS}

% Title page details: 
\title{Assignment 7} 
\author{Manikanta Uppulapu}
\date{\today}
% \logo{\large \LaTeX{}}


\begin{document}

% Title page frame
\begin{frame}
    \titlepage 
\end{frame}

% Remove logo from the next slides
\logo{}


% Outline frame
\begin{frame}{Outline}
    \tableofcontents
\end{frame}



\section{Problem Statement}
\begin{frame}{Problem Statement}
    \begin{block}{13.2 Q5 [NCERT 12] }A die marked $1, 2, 3$ in red and $4, 5, 6$ in green is tossed. Let $A$ be the event, 'the number is even,' and $B$ be the event, 'the number is red'. Are $A$ and $B$ independent?
    \end{block}
\end{frame}



\section{Given}
\begin{frame}{Given}
Let's denote the situation of the problem by random variable $X$  such that $X\in \cbrak{0,1,2,3}$\\
where,\\
\begin{table}[ht!]
    \centering
    \input{table}
    \caption{}
\label{table:table1}
\end{table}

  
\end{frame}

\begin{frame}{Probabilities}
\begin{table}[ht!]
    \centering
    \input{table2}
    \caption{Probability values}
\label{table:table1}
\end{table}

\end{frame}

\section{Solution}
\begin{frame}{Solution}
\begin{exampleblock}{Theorem}
        Two sets, say A and B, are independent if $P(AB) = P(A)P(B)$
   \end{exampleblock}
   \begin{block}{Conclusion}
        $\Pr(X=1)\Pr(X=3) = \frac{1}{2} \times \frac{1}{2} =   \frac{1}{4} \neq \frac{1}{6} \neq \Pr(X=1,3)$.\\
        Therefore , the two sets are not Independent.
    \end{block}

\end{frame}

\end{document}
